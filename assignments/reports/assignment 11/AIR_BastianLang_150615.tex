%%%%%%%%%%%%%%%%%%%%%%%%%%%%%%%%%%%%%%%%%
% Short Sectioned Assignment
% LaTeX Template
% Version 1.0 (5/5/12)
%
% This template has been downloaded from:
% http://www.LaTeXTemplates.com
%
% Original author:
% Frits Wenneker (http://www.howtotex.com)
%
% License:
% CC BY-NC-SA 3.0 (http://creativecommons.org/licenses/by-nc-sa/3.0/)
%
%%%%%%%%%%%%%%%%%%%%%%%%%%%%%%%%%%%%%%%%%

%----------------------------------------------------------------------------------------
%	PACKAGES AND OTHER DOCUMENT CONFIGURATIONS
%----------------------------------------------------------------------------------------

\documentclass[paper=a4, fontsize=11pt]{scrartcl} % A4 paper and 11pt font size

\usepackage[T1]{fontenc} % Use 8-bit encoding that has 256 glyphs
\usepackage{fourier} % Use the Adobe Utopia font for the document - comment this line to return to the LaTeX default
\usepackage[english]{babel} % English language/hyphenation
\usepackage{amsmath,amsfonts,amsthm} % Math packages

\usepackage{graphicx}
\usepackage{float}

\usepackage{sectsty} % Allows customizing section commands
\allsectionsfont{\normalfont\scshape} % Make all sections centered, the default font and small caps

\usepackage{fancyhdr} % Custom headers and footers
\pagestyle{fancyplain} % Makes all pages in the document conform to the custom headers and footers
\fancyhead{} % No page header - if you want one, create it in the same way as the footers below
\fancyfoot[L]{} % Empty left footer
\fancyfoot[C]{} % Empty center footer
\fancyfoot[R]{\thepage} % Page numbering for right footer
\renewcommand{\headrulewidth}{0pt} % Remove header underlines
\renewcommand{\footrulewidth}{0pt} % Remove footer underlines
\setlength{\headheight}{13.6pt} % Customize the height of the header

\numberwithin{equation}{section} % Number equations within sections (i.e. 1.1, 1.2, 2.1, 2.2 instead of 1, 2, 3, 4)
\numberwithin{figure}{section} % Number figures within sections (i.e. 1.1, 1.2, 2.1, 2.2 instead of 1, 2, 3, 4)
\numberwithin{table}{section} % Number tables within sections (i.e. 1.1, 1.2, 2.1, 2.2 instead of 1, 2, 3, 4)

\setlength\parindent{0pt} % Removes all indentation from paragraphs - comment this line for an assignment with lots of text

%----------------------------------------------------------------------------------------
%	TITLE SECTION
%----------------------------------------------------------------------------------------

\newcommand{\horrule}[1]{\rule{\linewidth}{#1}} % Create horizontal rule command with 1 argument of height

\title{	
\normalfont \normalsize 
\textsc{BRSU} \\ [25pt] % Your university, school and/or department name(s)
\horrule{0.5pt} \\[0.4cm] % Thin top horizontal rule
\huge Homework for Artificial Intelligence for Robotics\\Assignment 11 \\ % The assignment title
\horrule{2pt} \\[0.5cm] % Thick bottom horizontal rule
}

\author{Bastian Lang} % Your name

\date{\normalsize\today} % Today's date or a custom date

\begin{document}

\maketitle % Print the title


\section{Task 1}
Let A, B and C be propositional formulas such that A and B entails C (A $\land$ B $\vDash$ C):
\[A\land B\to C\]
Then, is $(A\to C) \lor (B \to C)$ always true?\\
\begin{table}[h]
\begin{tabular}{|c|c|c|c|c|c|c|c|}
\hline
A&B&C&$A\land B$&$A\land B\vDash C$ & $A\to C$& $B\to C$&$(A\to C)\lor (B\to C)$\\
\hline
+&+&+&+&+&+&+&+\\
+&+&-&+&-&-&-&-\\
+&-&+&-&+&+&+&+\\
+&-&-&-&+&-&+&+\\
\hline
-&+&+&-&+&+&+&+\\
-&+&-&-&+&+&-&+\\
-&-&+&-&+&+&+&+\\
-&-&-&-&+&+&+&+\\
\hline
\end{tabular}
\caption{Truth table for exercise 1}
\label{tab:ex1}
\end{table}

$A\to C$ is only true iff $\neg A \lor C$ is true. Similar $(A\land B)\to C$ is true iff $\neg (A\land B)\lor C$ is true and $B\to C$ is true iff $\neg B\lor C$ is true.\\
Using a truth table (table \ref{tab:ex1}) one can see that the term $(A\to C) \lor (B \to C)$ is true whenever $A\land B\to C$ is true.\\\qed

\section{Task 2}
\textbf{Prove the following:}\vspace{5mm}\\
(a) \hspace{10mm}$\neg P\land\neg Q\Leftrightarrow\neg(P\lor Q)$\\
(b) \hspace{10mm}$\neg (P\land Q) \Leftrightarrow\neg P\lor\neg Q$\\
(c) \hspace{10mm}$P\lor (P\land Q)\Leftrightarrow P$\\

\textbf{(a)}\\
$\neg P\land\neg Q\Leftrightarrow\neg(P\lor Q)$\\

According to the truth table \ref{tab:ex2a} this holds true.\\\qed

\begin{table}[h]
\begin{tabular}{|c|c|c|c|c|}
\hline
P&Q&$\neg P\land \neg Q$&$P\lor Q$&$\neg P\land \neg Q\Leftrightarrow\neg (P\lor Q)$\\
\hline
+&+&-&+&+\\
+&-&-&+&+\\
-&+&-&+&+\\
-&-&+&-&+\\
\hline
\end{tabular}
\caption{Truth table for exercise 2(a)}
\label{tab:ex2a}
\end{table}

\textbf{(b)}\\
$\neg (P\land Q) \Leftrightarrow\neg P\lor\neg Q$\\
According to the truth table \ref{tab:ex2b} this holds true.\\\qed

\begin{table}[h]
\begin{tabular}{|c|c|c|c|c|}
\hline
P&Q&$\neg (P\land Q)$&$\neg P\lor \neg Q$&$\neg (P\land Q)\Leftrightarrow\neg P\lor \neg Q$\\
\hline
+&+&-&-&+\\
+&-&+&+&+\\
-&+&+&+&+\\
-&-&+&+&+\\
\hline
\end{tabular}
\caption{Truth table for exercise 2(b)}
\label{tab:ex2b}
\end{table}

\textbf{(c)}\\
$P\lor (P\land Q)\Leftrightarrow P$\\
According to the truth table \ref{tab:ex2c} this holds true.\\\qed

\begin{table}[h]
\begin{tabular}{|c|c|c|c|c|}
\hline
P&Q&$P\land Q$&$P\lor (P\land Q)$&$P\lor (P\land Q)\Leftrightarrow P$\\
\hline
+&+&+&+&+\\
+&-&-&+&+\\
-&+&-&-&+\\
-&-&-&-&+\\
\hline
\end{tabular}
\caption{Truth table for exercise 2(c)}
\label{tab:ex2c}
\end{table}
\newpage
\section{Task 3}
\textbf{Proove that:}\\
\[P\to Q \vDash\neg Q\to\neg P\]
The term: \[P\to Q \vDash\neg Q\to\neg P\]
can be rephrased to:
\[(P\to Q) \to(\neg Q\to\neg P)\equiv \neg (P\to Q) \lor(\neg Q\to\neg P)\]

It is \[P\to Q\equiv \neg P\lor Q \]
and:
\[\neg (P\to Q)\equiv \neg (\neg P\lor Q)\equiv (P\land\neg Q)\]

It is:
\[\neg Q\to \neg P\equiv Q\lor\neg P\]\vspace{5mm}
Therefore:
\[(P\to Q \vDash\neg Q\to\neg P) \equiv ((P\land\neg Q)\lor (Q\lor\neg P))\]

Truth table \ref{tab:ex3} then shows that this holds true.\\\qed

\begin{table}[h]
\begin{tabular}{|c|c|c|c|c|c|}
\hline
P&Q&$\neg P\lor Q$&$P\land\neg Q$&$Q\lor\neg P$&$(P\land\neg Q)\lor (Q\lor\neg P)$\\
\hline
+&+&+&-&+&+\\
+&-&-&+&-&+\\
-&+&+&-&+&+\\
-&-&+&-&+&+\\
\hline
\end{tabular}
\caption{Truth table for exercise 3}
\label{tab:ex3}
\end{table}

\section{Task 4}
\textit{What is Inference? What do we use it for in Artificial Intelligence? Give some examples.}\\

Inference is "The act or process of deriving logical conclusions from premises known or assumed to be true." (http://www.thefreedictionary.com/inference).\\

According to "The Role Of Logic In Artificial Intelligence" by Robert C. Moore in 1984 we can use Logic for three things:\\
(1) as an analytical tool\\
(2) as a knowledge representation formalism and method of reasoning\\
(3) as a programming language\\

In the Wumpus-World for example given an agents observations in one cell the agent can make predictions about neighbouring cells. A stench observed would for example lead to the conclusion that there is a wumpus in one of the neighbouring cells.\\



\section{Task 5}
Express the following sentences in propositional logic and determine if they are true or
not:
\begin{itemize}
\item Either John isn't stupid and he is lazy, or he's stupid.\\
John is stupid.\\
Therefore, John isn't lazy.
\item The butler and the cook are not both innocent.\\
Either the butler is lying or the cook is innocent.\\
Therefore, the butler is either lying or guilty.
\end{itemize}

\subsection{}
(1) Either John isn't stupid and he is lazy, or he's stupid.\\
(2) John is stupid.\\
(3) Therefore, John isn't lazy.\\

\textbf{Defining}:\\
$A :=$ John is stupid\\
$B :=$ John is lazy\\

Then (1),(2) and (3) can be written as:\\
(1) $(\neg A\land B)\lor A$\\
(2) $A$\\
(3) $(1)\land (2)\to\neg B\equiv \neg ((1)\land (2))\lor\neg B\equiv (\neg (1) \lor\neg (2))\lor\neg B)$\\
Truth table \ref{tab:ex5.1} shows that statement (3) does not hold if John is stupid and lazy. (1) and (2) would be true, but John still would be lazy.\\\qed
\begin{table}[h]
\begin{tabular}{|c|c|c|c|}
\hline
(2): A&B&(1): $(\neg A\land B)\lor A$&(3): $\neg (1) \lor\neg (2))\lor\neg B$\\
\hline
+&+&+&-\\
+&-&+&+\\
-&+&+&+\\
-&-&-&+\\
\hline
\end{tabular}
\caption{Truth table for exercise 5.1}
\label{tab:ex5.1}
\end{table}

\subsection{}
(1) The butler and the cook are not both innocent.\\
(2) Either the butler is lying or the cook is innocent.\\
(3) Therefore, the butler is either lying or guilty.\\

\textbf{Defining:}\\
$B :=$ Butler is innocent\\
$C :=$ Cook is innocent\\
$B_L :=$ Butler is lying\\

(1),(2) and (3) can be written as:\\
(1) $\neg (B\land C)\equiv (\neg B\lor\neg C)$\\
(2) $B_L\lor C$\\
(3) $((1)\land (2))\to (B_L\lor\neg B)\equiv (\neg (1) \lor\neg (2)\lor B_L\lor \neg B)$\\

Truth table \ref{tab:ex5.2} shows that statement (3) holds for every configuration of B, $B_L$ and C.\\\qed


\begin{table}[h]
\begin{tabular}{|c|c|c|c|c|c|}
\hline
B&$B_L$&C&(1): $\neg B\lor\neg C$&(2): $B_L\lor C$&(3): $\neg (1) \lor\neg (2)\lor B_L\lor \neg B$\\
\hline
+&+&+&-&+&+\\
+&+&-&+&+&+\\
+&-&+&-&+&+\\
+&-&-&+&-&+\\
-&+&+&+&+&+\\
-&+&-&+&+&+\\
-&-&+&+&+&+\\
-&-&-&+&-&+\\

\hline
\end{tabular}
\caption{Truth table for exercise 5.2}
\label{tab:ex5.2}
\end{table}

\section{Task 6}
Three students are suspects of plagiarism, but the teachers do not know which student did
the original work and which students plagiarized. The students have been interrogated
separately and their answers are:
\begin{itemize}
\item Student A says that if Student B did the original homework, then also Student C
helped him do the original work.
\item Student B says that if Student A did not do the original homework, then also Student
C did not do the original work.
\item Student C says that Student A did the original work, and Student B did not do the
original work.
\end{itemize}

Then:\\
(a) Are all declarations compatible between each other?\\
(b) If all three students plagiarized each other, which student is lying?\\
(c) If everyone is telling the truth, then who did the original work and who plagiarized?\\
(d) If everyone is lying, then who did the original work and who plagiarized?\\

\textbf{Define:}\\
$A:=$ A did the original work\\
$B:=$ B did the original work\\
$C:=$ C did the original work\\

Answers:\\
(1) $B\to C\equiv\neg B\lor C$\\
(2) $\neg A\to\neg C\equiv A\lor\neg C$\\
(3) $A\land\neg B$\\

According to the truth table \ref{tab:ex6}:\\
(a) there is a configuration that all answers could be true when A and C did the original work. \\
(b) ff all students plagiarized, then C would be lying (last line of table).\\
(c) there are two possible configurations for all telling the truth. In both cases A and B did not plagiarize, but C could have done both. \\
(d) it is not possible that all students are lying (last column of table).

\begin{table}[h]
\begin{tabular}{|c|c|c|c|c|c|c|c|}
\hline
A&B&C&(1)&(2)&(3)&$(1)\land (2)\land (3)$&$\neg (1)\land\neg (2)\land\neg (3)$\\
\hline
+&+&+&+&+&-&-&-\\
+&+&-&-&+&-&-&-\\
+&-&+&+&+&+&+&-\\
+&-&-&+&+&+&+&-\\
-&+&+&+&-&-&-&-\\
-&+&-&-&+&-&-&-\\
-&-&+&+&-&-&-&-\\
-&-&-&+&+&-&-&-\\

\hline
\end{tabular}
\caption{Truth table for exercise 6}
\label{tab:ex6}
\end{table}
\end{document}