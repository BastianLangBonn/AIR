%%%%%%%%%%%%%%%%%%%%%%%%%%%%%%%%%%%%%%%%%
% Short Sectioned Assignment
% LaTeX Template
% Version 1.0 (5/5/12)
%
% This template has been downloaded from:
% http://www.LaTeXTemplates.com
%
% Original author:
% Frits Wenneker (http://www.howtotex.com)
%
% License:
% CC BY-NC-SA 3.0 (http://creativecommons.org/licenses/by-nc-sa/3.0/)
%
%%%%%%%%%%%%%%%%%%%%%%%%%%%%%%%%%%%%%%%%%

%----------------------------------------------------------------------------------------
%	PACKAGES AND OTHER DOCUMENT CONFIGURATIONS
%----------------------------------------------------------------------------------------

\documentclass[paper=a4, fontsize=11pt]{scrartcl} % A4 paper and 11pt font size

\usepackage[T1]{fontenc} % Use 8-bit encoding that has 256 glyphs
\usepackage{fourier} % Use the Adobe Utopia font for the document - comment this line to return to the LaTeX default
\usepackage[english]{babel} % English language/hyphenation
\usepackage{amsmath,amsfonts,amsthm} % Math packages

\usepackage{graphicx}
\usepackage{float}

\usepackage{sectsty} % Allows customizing section commands
\allsectionsfont{\normalfont\scshape} % Make all sections centered, the default font and small caps

\usepackage{fancyhdr} % Custom headers and footers
\pagestyle{fancyplain} % Makes all pages in the document conform to the custom headers and footers
\fancyhead{} % No page header - if you want one, create it in the same way as the footers below
\fancyfoot[L]{} % Empty left footer
\fancyfoot[C]{} % Empty center footer
\fancyfoot[R]{\thepage} % Page numbering for right footer
\renewcommand{\headrulewidth}{0pt} % Remove header underlines
\renewcommand{\footrulewidth}{0pt} % Remove footer underlines
\setlength{\headheight}{13.6pt} % Customize the height of the header

\numberwithin{equation}{section} % Number equations within sections (i.e. 1.1, 1.2, 2.1, 2.2 instead of 1, 2, 3, 4)
\numberwithin{figure}{section} % Number figures within sections (i.e. 1.1, 1.2, 2.1, 2.2 instead of 1, 2, 3, 4)
\numberwithin{table}{section} % Number tables within sections (i.e. 1.1, 1.2, 2.1, 2.2 instead of 1, 2, 3, 4)

\setlength\parindent{0pt} % Removes all indentation from paragraphs - comment this line for an assignment with lots of text

%----------------------------------------------------------------------------------------
%	TITLE SECTION
%----------------------------------------------------------------------------------------

\newcommand{\horrule}[1]{\rule{\linewidth}{#1}} % Create horizontal rule command with 1 argument of height

\title{	
\normalfont \normalsize 
\textsc{BRSU} \\ [25pt] % Your university, school and/or department name(s)
\horrule{0.5pt} \\[0.4cm] % Thin top horizontal rule
\huge Homework for Artificial Intelligence for Robotics\\Assignment 11 \\ % The assignment title
\horrule{2pt} \\[0.5cm] % Thick bottom horizontal rule
}

\author{Bastian Lang} % Your name

\date{\normalsize\today} % Today's date or a custom date

\begin{document}

\maketitle % Print the title


\section{Task 1}
\textit{What are the performance differences between Backward and Forward Chaining?}

Forward Chaining is data-driven and Backward Chaining is goal-driven. The complexity of Backward Chaining can be much less than linear in the size of KB.

\section{Task 2}
\textit{Can the following sentence be expressed in propositional logic, considering all real numbers?}
\begin{itemize}
\item \textit{Negative numbers don't have square roots.}
\item \textit{Every positive number has exactly two square roots.}
\end{itemize}

To make a statement for every positive or negative number in propositional logic one would in theory have to enumerate every possible number and make a statement. Because the real numbers are non-denumerable it is not possible to express the above sentences using propositional logic.

\section{Task 3}
\textit{Represent the following sentences in first-order logic, using a consistent vocabulary (which
you must define):}
\begin{itemize}
\item \textit{(a) A person born in the UK, each of whose parents is a UK citizen or a UK resident, is
a UK citizen by birth.}
\item \textit{(b) A person born outside the UK, one of whose parents is a UK citizen by birth, is a
UK citizen by descent.}
\item \textit{(c) Politicians can fool some of the people all of the time, and they can fool all of the
people some of the time, but they can not fool all of the people all of the time.}
\item \textit{(d) There is a barber who shaves all men in the town who do not shave themselves.}
\item \textit{(e) The best score in Greek is always higher than the best score in French.}
\end{itemize}

\subsection{(a)}
Let $P_i$ denote a person, $born\textunderscore in\textunderscore uk(P_i)$ evaluates true iff person $P_i$ is born in the UK, $citizen\textunderscore of\textunderscore uk(P_i)$ iff a person is citizen of the UK, $resident\textunderscore of\textunderscore uk(P_i)$ iff a person is a resident of the UK and $parent(P_i, P_j)$ iff $P_i$ is parent of $P_j$.

Then (a) can be expressed as:\\
$\forall P_1, P_2, P_3:$\\
$(born\textunderscore in\textunderscore uk(P_3) \land parent(P_1,P_3)\land parent(P_2, P_3)$\\
$\land (citizen\textunderscore of\textunderscore uk(P1)\lor resident\textunderscore of\textunderscore uk(P1))$\\
$\land (citizen\textunderscore of\textunderscore uk(P_2)\lor resident\textunderscore of\textunderscore uk(P_2))$\\
$\to citizen\textunderscore of\textunderscore uk\textunderscore by\textunderscore birth(P_3))$

\subsection{(b)}
$\forall P_1, P_2, P_3:$\\
$(\neg born\textunderscore in\textunderscore uk(P_3)\land parent(P_1, P_3)\land parent(P_2, P_3)$\\
$\land (citizen\textunderscore of\textunderscore uk\textunderscore by\textunderscore birth(P_1) \lor citizen\textunderscore of\textunderscore uk\textunderscore by\textunderscore birth(P_2)))$\\
$\to citizen\textunderscore of\textunderscore uk\textunderscore by\textunderscore descent(P_3))$

\subsection{(c)}

\subsection{(d)}
Let Barber(x) be true iff x is a barber, inTown(x) if x lives in town, man(x) iff x is a man and shaves(x,y) iff x shaves y. Then:\\

$\exists x: barber(x)\land inTown(x)\land \forall y: man(y) \land inTown(y)\land ((shaves(x,y) \land \neg shaves(y,y))\lor shaves(y,y))$ 

\subsection{(e)}


\section{Task 4}
\textit{PROLOG is a programming language commonly used for logic. It uses first-order logic.\\
Learn some basic PROLOG 1 and write axioms describing the following relationships:\\
Child, Grandchild, GreatGrandparent, Brother, Sister, Daughter, Son, Aunt, Uncle}\\

parent(john, mary).\\
parent(mary, william).\\
parent(mary, emma).\\
parent(william, sophia).\\
parent(william, ryan).\\
parent(william, jacob).\\
parent(emma, joseph).\\
parent(emma, olivia).\\
parent(emma, emily).\\
female(mary).\\
female(sophia).\\
female(emma).\\
female(olivia).\\
female(emily).\\
male(john).\\
male(william).\\
male(ryan).\\
male(jacob).\\
male(joseph).\\
child(X,Y) :- parent(Y,X).\\
grandparent(X,Y) :- parent(X,Z), parent(Z,Y).\\
brother(X,Y) :- parent(Z,X), parent(Z,Y), male(X), X \textbackslash =Y.\\
sister(X,Y) :- parent(Z,X), parent(Z,Y), female(X), X \textbackslash = Y.\\
uncle(X,Y) :- brother(X,Z), child(Y,Z).\\
aunt(X,Y) :- sister(X,Z), child(Y,Z).\\
daughter(X,Y) :- child(X,Y), female(Y).\\
son(X,Y) :- child(X,Y), male(X).\\
grandchild(X,Y) :- child(X,Z), child(Z,Y).\\
greatgrandparent(X,Y) :- parent(X,Z), parent(Z,A), parent(A,Y).\\
\end{document}